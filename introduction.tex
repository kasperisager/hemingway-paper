\section{Introduction}
\label{introduction}

An increasingly popular approach for tackling similarity search in high-dimensional datasets is the so-called \textit{locality-sensitive hashing} (\textit{LSH}) technique, first presented by Piotr Indyk and Rajeev Motwani \cite{DBLP:conf/stoc/IndykM98}. LSH has therefore already found itself useful in a wide range of practical applications:

\begin{itemize}
  \item Nearest neighbour search
  \item Near-duplicate detection
  \item Hierarchical clustering
  \item Genome-wide association study
  \item Image and Audio similarity identification
  \item Audio, video, human fingerprinting
\end{itemize}

The basic idea of LSH is to hash items to buckets in a way that provides a higher probability of similar items being hashed to the same bucket than dissimilar items. Items that then hash to the same bucket despite them being dissimilar are \textit{false positives}. On the other hand, similar items that never hash to the same bucket are \textit{false negatives} \cite[p. 88]{DBLP:books/cu/LeskovecRU14}. While false positives have no effect on the precision of queries, false negatives may cause the algorithm to never consider items that are in fact the most similar to a query item.

A recent paper by Rasmus Pagh \cite{DBLP:journals/corr/Pagh15} proposes an LSH scheme for Hamming space that completely does away with false negatives at a cost in efficiency. The purpose of our paper is to provide a generalised implementation of this LSH scheme, named \textit{covering LSH}, and compare it with classic LSH. Generalised mean that it will provide library like funcionalities where the user can freely specify settings, data and avoid limitations on data size and dimensionality.

\paragraph{Organisation} This paper is organised as follows: In section \ref{background} we provide the background for classic LSH, including how it works and its inherent drawbacks, and outline the covering LSH scheme and the gurantees that it provides. In section \ref{implementation} we describe our implementation of the two LSH schemes. In section \ref{evaluation} an experimental evaluation based on real data is made after which we compare the two LSH schemes.

\section{Introduction}
\label{introduction}

An increasingly popular approach for tackling similarity search in high-dimensional datasets is the so-called \textit{locality-sensitive hashing} (\textit{LSH}) technique, first presented by Piotr Indyk and Rajeev Motwani \cite{DBLP:conf/stoc/IndykM98}. LSH has therefore already found itself useful in a wide range of practical applications:

\begin{itemize}
  \item Nearest neighbour search
  \item Near-duplicate detection
  \item Hierarchical clustering
  \item Genome-wide association study
  \item Image and audio similarity identification
  \item Human fingerprint recognition
\end{itemize}

The basic idea of LSH is to hash items to \textit{buckets} in a way that provides a higher probability of similar items being hashed to the same bucket than dissimilar items. Items that then hash to the same bucket despite them being dissimilar are \textit{false positives}. On the other hand, similar items that never hash to the same bucket are \textit{false negatives} \cite[p. 88]{DBLP:books/cu/LeskovecRU14}. While false positives have no effect on the precision of queries, false negatives may cause the algorithm to never consider items that are in fact the most similar to a query item. The latter becomes a problem in settings that require exact rather than probabilistic guarantees of returning a nearest neighbour, as is the case in for example human fingerprint recognition.

A recent paper by Rasmus Pagh \cite{DBLP:journals/corr/Pagh15} proposes an LSH scheme that completely does away with false negatives at a cost in efficiency. The purpose of our paper is to compare this LSH scheme, named \textit{covering LSH}, with classic LSH on a number of different metrics such as query throughput and filtering efficiency.

\paragraph{Organisation} This paper is organised as follows: In section \ref{background} we provide the background for classic LSH and outline the covering LSH scheme and the guarantees that it provides. In section \ref{implementation} we describe our implementation of the two LSH schemes. In section \ref{evaluation} an experimental evaluation based on a real dataset is made after which we compare the two LSH schemes. Finally, in section \ref{conclusion} we give our parting thoughts on the covering LSH scheme and the cost of the exact guarantees that it provides.

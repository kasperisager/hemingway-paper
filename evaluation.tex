\section{Evaluation}
 \subsection{Data set}
 Our dataset is the set called \textit{ANN\_IFT1M}\footnote{\url{http://corpus-texmex.irisa.fr/}}. Which is a set of 1,000,000 vectors
 with dimensionality 128. Vectors are stored as raw little endians.
 This dataset is created especially for measurements of the approximate nearest 
neighbour searches.  
 \subsection{Metrics}
 \paragraph{Recall} Recall in information retrieval is the fraction of the data that are relevant to the query that are successfully retrieved.
 \paragraph{Precision} Precision is the fraction of retrieved data that are relevant to the query. Calculated by given formula:
 \paragraph{Time} We have three different speed metrics to compare:
 \begin{itemize}
  \item Construction time given in miliseconds. 
  \item Insertion time given in nanoseconds.
  \item Query time given in seconds.
\end{itemize}
 \paragraph{Memory consumptionl} Total amount of memory needed to create data structure and query it. Given in megabytes and gigabytes.
 \subsection{Benchmark} For measuring our metrics we are using library called 
 \textit{hayai}\footnote{\url{https://github.com/nickbruun/hayai}}, small C++ framework for writting benchmarks. It provides toolkit for measuring time, multiple runs and iterations.
  \subsection{Test environment} 
  Machine spec and software versions. 
\label{evaluation}

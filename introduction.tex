\section{Introduction}
\label{introduction}

An increasingly popular approach for tackling similarity search in high-dimensional datasets is the so-called \textit{locality-sensitive hashing} (\textit{LSH}) technique, first presented by Piotr Indyk and Rajeev Motwani \cite{DBLP:conf/stoc/IndykM98}. The basic idea of LSH is to hash items to buckets in a way that provides a higher probability of similar items being hashed to the same bucket than dissimilar items. Items that then hash to the same bucket despite them being dissimilar are \textit{false positives}. On the other hand, similar items that never hash to the same bucket are \textit{false negatives}. While false positives have no effect on the precision of queries, false negatives may cause the algorithm to never consider items that are in fact the most similar to a query item.

A recent paper by Rasmus Pagh \cite{DBLP:journals/corr/Pagh15} proposes an LSH scheme for Hamming space that completely does away with false negatives at a cost in efficiency. The purpose of our paper is to provide a generalised implementation of this LSH scheme and compare it with classic LSH.

\textbf{Organisation} This paper is organised as follows: In section \ref{background} we provide the background for classic LSH, including how it works and its inherit drawbacks, and outline the covering LSH scheme and the gurantees that it provides. In section \ref{implementation} we describe our implementation of the two LSH schemes. In section \ref{evaluation} an experimental evaluation based on real data is made after which we compare the two LSH schemes.

\begin{abstract}
One of the latest developments in the area of locality-sensitive hashing (LSH) is the so-called \textit{covering LSH} scheme. This scheme solves one of the primary drawbacks of classic locality-sensitive hashing, namely \textit{false negatives}, i.e. items that never collide with a query item despite them being similar. Covering LSH is therefore of interest in settings that require exact guarantees of producing a nearest neighbour, making classic LSH unsuitable as it only provides probabilistic guarantees. Given that the covering LSH scheme is still relatively new, it still lacks, to the best of our knowledge, a proper general-purpose implementation and associated evaluation thereof.

By implementing both the covering and the classic LSH schemes and benchmarking them on a real-world dataset designed for the evaluation of approximate nearest neighbour search algorithms, we show that within the same time and memory bounds, covering LSH is able to not only match but in fact outperform the filtering efficiency of classic LSH. This does however come at a negligible cost in insertion time and with an additional implementation specific memory overhead.
\end{abstract}

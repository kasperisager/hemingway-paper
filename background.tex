\section{Background}
\label{background}

As touched upon in section \ref{introduction}, locality-sensitive hashing tackles the problem of similarity search in high-dimensional datasets by relaxing the requirement of finding exact nearest neighbours. Instead, LSH is used as an \textit{approximate} nearest neighbour algorithm, which inherently implies only probabilistic guarantees of finding a nearest neighbour. In general, LSH schemes act as randomized filters that attempt to reduce an input set to a subset of candidates for a given query item.

\subsection{Classic LSH}
\label{background-classic-lsh}

\subsection{Covering LSH}
\label{background-covering-lsh}

\section{Background}
\label{background}

As touched upon in section \ref{introduction}, locality-sensitive hashing tackles the problem of similarity search in high-dimensional datasets by relaxing the requirement of finding exact nearest neighbours. Instead, LSH is used as an \textit{approximate} nearest neighbour algorithm, which inherently implies only probabilistic guarantees of finding a nearest neighbour. In general, LSH schemes act as randomized filters that attempt to reduce an input set to a subset of candidates for a given query item.

\subsection{Classic LSH}
\label{background-classic-lsh}

The classic LSH scheme for Hamming space uses a random bit sampling approach for picking a number of components from an input vector. This sample is then used as the key in a map structure for locating a bucket that the vector should be stored in. When looking for the nearest neighbour of a query vector, this sampling is repeated and the candidate vectors are those found in the bucket that the sample maps to.

\begin{example}
\label{example-classic-sampling}
Given input vector $v = 1101$, we randomly chose to sample component 1 and 3, giving us the key $v' = 10$. We then proceed to update our map structure with an entry for this key: $10 \rightarrow \{1101\}$

Given another input vector $u = 0110$, we again sample component 1 and 3, giving us the key $u' = 00$. We then add another entry to our map: $10 \rightarrow \{1101\}, 01 \rightarrow \{0110\}$.

Given a query vector $q = 1001$, we once again sample component 1 and 3, giving us the key $q' = 10$. We then look up this key in our map and receive the following set of candidates: $\{1101\}$.
\end{example}

As can be seen in example \ref{example-classic-sampling}, $v$ and $u$ are not particularly similar as they only share a single component; by the sampling 1 and 3 they therefore do not map to the same bucket. However, $v$ and $q$ are almost identical as they share all but one component; the sampling 1 and 3 therefore maps them to the same bucket, albeit by chance. We have effectively reduced the set of potential candidates to half of the items in the original input set.

\subsection{Covering LSH}
\label{background-covering-lsh}

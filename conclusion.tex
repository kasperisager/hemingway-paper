\section{Conclusion}
\label{conclusion}

We have in this paper shown how the proposed covering LSH scheme performs compared to the classic LSH scheme when both are implemented as part of a general-purpose C++ library. Based on our evaluation, we can conclude that the query throughput of covering LSH outperforms that of classic LSH given the same time and memory bounds. This as a result of an improved distribution of items in buckets across partitions, which in turn reduces the average size of candidate sets for query items. The improved bucket distribution does however come at a cost in the form of a decrease in insertion time and an additional memory overhead, both caused by the increased number of buckets.

In addition, our results verify the claim made in \cite{DBLP:journals/corr/Pagh15} that the efficiency of the covering LSH scheme matches that of the classic LSH scheme of \cite{DBLP:conf/stoc/IndykM98} in the case that $cr = \log n$. As stated earlier, the efficiency of our implementation in fact surpasses that of the classic LSH scheme while maintaining the exact guarantees of producing a nearest neighbour.
